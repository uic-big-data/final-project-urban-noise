
\section{Related work}
Our application will be based on the research paper ~ \cite{4}, which presents the process used to collect this data. SONYC has developed an acoustic sensor with high quality and low production cost to monitor the noise pollution levels across the city. The sensors follow DCASE (Detection and classification of acoustic scenes and events) to eliminate discrepancy. There are various other datasets like UrbanSound, UrbanSound8k that address this particular problem but have limited spacial and temporal data points. A VGGish model has been developed and trained using stochastic gradient descent to minimize cross-entropy loss. To eliminate over-fitting early stopping on validation set has been implemented. The overall AUPRC achieved by this model is 0.62 and 0.76 on different level classes, which performed poorly on music and non-machinery impact sounds.

According to [2] a supervised learning methodology is applied to real-life high quality recordings of 3-5 minutes with very little noise of 15 different acoustic scenes (lakeside beach, bus, cafe/restaurant, car, city
center, forest path, grocery store, home, library, metro station,
office, urban park, residential area, train, and tram) and two common enviornment areas (outdoor - residential areas and indoor - home). A mel frequency cepstral coefficient (MFCC) and Gaussian mixture model (GMM) was trained using expectation maximization algorithm. The overall accuracy achieved by the model is 72.5 \% ranging from 13.9\% for parks to 98.6\% for office spaces.

The partitioning of the data was done based on the location
of the original recordings. All segments obtained from the
same original recording were included into a single subset
- either development or evaluation. This is a very important
detail that is sometimes neglected, and failing to recognize
it results in overestimating the system performance, as the
classification systems are capable of learning the location-
specific acoustic conditions instead of the intended general
audio scene properties,. The phenomenon is similar to the
”album effect” encountered in music information retrieval, that
has been noticed and is usually accounted for when setting up
experiments. The cross-validation setup provided with the
database consists of four folds distributing the 78 segments
available in the development set based on location.
